
\documentclass[11pt, a4paper,twocolumn]{article}

%\usepackage[margin=1in]{geometry}
\usepackage[T1]{fontenc}
\usepackage{verbatim}

\usepackage{ifpdf}
\ifpdf 
    \usepackage[pdftex]{graphicx}   % to include graphics
    \pdfcompresslevel=9 
    \usepackage[pdftex,     % sets up hyperref to use pdftex driver
            plainpages=false,   % allows page i and 1 to exist in the same document
            breaklinks=true,    % link texts can be broken at the end of line
            colorlinks=true,
            pdftitle=My Document
            pdfauthor=Robin Solignac
           ]{hyperref} 
    \usepackage{thumbpdf}
\else 
    \usepackage{graphicx}       % to include graphics
    \usepackage{hyperref}       % to simplify the use of \href
\fi 

\title{Student Seminar: Exploiting Two Factor Authentication of Android and IOS}
\author{Robin Solignac 235020}
\date{\today}

\begin{document}
\maketitle
\begin{comment}
\begin{abstract}
Applications  which  can  be  used  in  different  platforms  now  use  two
factor authentication (2FA) to allow users to conveniently switch from one platform to another.
For example, when a user tries to login his gmail, it is not enough to enter correct password 
(first factor), it is also necessary to enter a PIN which is received by an SMS (second factor).
The aim of this project explaining the attacks against 2FA in IOS and Android devices and 
what can be the solutions. 
\end{abstract}
\end{comment}

\section{Introduction: The current 2 Factors Authentification}

Two factor authentification (2FA) is a combinaison of 2 acces control requirement in 
order to make it more robust to attacker. to authenticate yourself to an online 
service you both need to provide something you know (your password) and 
something you have. 
A large majority of people in todays world has phone and a large share of 
them are smart, one of the most most used ``something you have''  is a phone or 
a smartphone

More than that, it's the fact that the service can through this phone
send you a message  who will not transit on your (maybe compromised) PC 
who is used. your smartphone is an out of band channel.

\begin{comment} But it's not the only fact of having a phone or a phonenumber who is 
used in pratice to authenticate but the fact that services can send message to it and that that 
user can get it without using the main (maybe compromised) communication 
channel. This is called an . 
\end{comment} 

the current 2FA scheme is the fllowing: after a succesful password authtification it send a One-
Time Passwords (OTP) to the smartphone of the user most of the time via SMS 
but also sometime via a dedicated app.

%\subsection{2FA: 2 Factors Authentification}
\section{Key concepts}
\subsection{Syncornisation}
Our use of informatic is mostly divided beetween our personnal computers 
and smartphones. for this reason software manufacturer have sometime decided to implement 
syncronisation processes beetween the 2 devices in order to make transition and 
general use of the 2 smoother, bluring the line beetween the two. 

3 exemple of such syncronisation who will be used in our attack are the 
following:
\begin{description}%[\IEEEsetlabelwidth{Google play remote install}]
  \item[Google play remote install] it's possible from a PC via the google play 
  website to remotly install an application on our phone if both are logged on 
  the same google account. the only thing appering on the phone afterward is the 
  application incon in the app tray and an notification saying ``<app\_name> as been succesfully installed'' 
  
  \item[Apple Continuity] On recent version of Ios and macOs you can enable this 
  setting to syncronise in clear, read and send your SMS from your mac.
  
  \item[Browser syncronisation] Allmost all of today's most popular internet browser 
  propose syncornization beetween theire mobile and desktop version logged under same user account.
  it will sync history, boommark and sometime currently open tabs 
\end{description}
\subsection{MitB: Man in the browser attack}
MitB is a type of attack who assume that the attacker has an entier control and view on the PC browser of the victim. Like a man in the middle attack, the attaker can see all data 
exchanged by the browser and server and can modify them (on the fly). it also can send and receive data in the name of the user. But unlike the former it has acces to these data before they encrypted (or after they are decryted) And has also modify browser related setting like bookmark and current open tabs URLs. In short powerfull man in the browser attack can remotelly perform the same actions has someone getting physical acces to the browser

there are different way to do a MitB attack, using malware infecting the whole system, by API hooking or via malicious plugin. TO COMPLETE ?
\section{2FA Attack on }
\subsection{Android}
The principle to the attack is to, via a MitB, install from the web play-store anapplication from with autorization to read SMS (autorisation confirmed from the browser 
too). Then when an SMS is received it is forward to the attacker, thus bypassing 
F2A. In order to succed the attacker need to pass two defense setup by Google.

\paragraph{Bypassing Google boncer} llzezeffe

\paragraph{Activate the app} When installed, an Android app can't be triggered by external event (such as RECEIVE\_SMS) until it has been explicitly open for the first time. So we need to trick the user to open the app from the phone. The 
first way is to give a clickbait title to the app so that the user will be tempted to open it when he sees the installation notification.The second is to trigger its opening from the mobile browser by cliking a link

\subsection{Ios}
Since 2015 Ios has forbid the application to read all 
notificatino or SMS without explicit autorization (or they will be rejected from entering the appstore). 
Since then, the previous attack does not work. 

However if the infected browser is on a Mac and Continity 
is activated on the Iphone. the browser can still has acces to to SMS in clear 
on the mac as soon as both are on the same LAN, which is likely to eventually occurs
as they belong to the same person. 

So under these pretty likely to appers conditons, the 2FA authentification can 
be very easily bypassed on 2FA by an MitB attack.


\section{Discutions} 
\subsection{Android}
\subsection{Ios}
\section{Conclusion}

%\bibliography{ref} 
%\bibliographystyle{ieeetr}

\end{document}  
