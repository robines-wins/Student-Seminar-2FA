
\documentclass[11pt]{article}

\usepackage{ifpdf}
\ifpdf 
    \usepackage[pdftex]{graphicx}   % to include graphics
    \pdfcompresslevel=9 
    \usepackage[pdftex,     % sets up hyperref to use pdftex driver
            plainpages=false,   % allows page i and 1 to exist in the same document
            breaklinks=true,    % link texts can be broken at the end of line
            colorlinks=true,
            pdftitle=My Document
            pdfauthor=Robin Solignac
           ]{hyperref} 
    \usepackage{thumbpdf}
\else 
    \usepackage{graphicx}       % to include graphics
    \usepackage{hyperref}       % to simplify the use of \href
\fi 

\title{Student Seminar: \\Exploiting Two Factor Authentication of Android and IOS}
\author{Robin Solignac 235020}
\date{\today}

\begin{document}
\maketitle
\begin{abstract}
  Applications  which  can  be  used  in  different  platforms  now  use  two
factor authentication (2FA) to allow users to conveniently switch from one platform to another.
For example, when a user tries to login his gmail, it is not enough to enter correct password (first
factor), it is also necessary to enter a PIN which is received by an SMS (second factor).  The aim
of this project explaining the attacks against 2FA in IOS and Android devices and what it can
be the solution. 
\end{abstract}
\section{Introduction: The 2 Factors Authentification}

%\subsection{2FA: 2 Factors Authentification}
\section{Key concepts}
\subsection{Smartphone and anywhere computing}
Today smartphone are more present than ever and a large majority of personn
\subsection{MitB: Man in the browser attack}
MitB is a type of attack who assume that the attacker has an entier control and 
view on the PC browser of the victim. Like a man in the middle attack, the attaker can see all data 
exchanged by the browser and server and can modify them (on the fly). it also can send and receive 
data in the name of the user. 
But unlike the former it has acces to these data before they encrypted (or after they are decryted) 
And has also modify browser related setting like bookmark and current open tabs 
URLs. In short power full man in the browser attack can remotelly perform the same action 
has someone getting physical acces to the browser
\paragraph{}

\section{2FA Attack on }
\subsection{Android}
\subsection{Ios}
\section{Discutions} 
\section{conclusion}

\end{document}  
