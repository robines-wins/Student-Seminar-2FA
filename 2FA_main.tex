
\documentclass[11pt, a4paper,twocolumn]{article}

%\usepackage[margin=1in]{geometry}
\usepackage[T1]{fontenc}
\usepackage{verbatim}

\usepackage{ifpdf}
\ifpdf 
    \usepackage[pdftex]{graphicx}   % to include graphics
    \pdfcompresslevel=9 
    \usepackage[pdftex,     % sets up hyperref to use pdftex driver
            plainpages=false,   % allows page i and 1 to exist in the same document
            breaklinks=true,    % link texts can be broken at the end of line
            colorlinks=true,
            pdftitle=My Document
            pdfauthor=Robin Solignac
           ]{hyperref} 
    \usepackage{thumbpdf}
\else 
    \usepackage{graphicx}       % to include graphics
    \usepackage{hyperref}       % to simplify the use of \href
\fi 

\title{Student Seminar: Exploiting Two Factor Authentication of Android and IOS}
\author{Robin Solignac 235020}
\date{\today}

\begin{document}
\maketitle
\begin{comment}
\begin{abstract}
Applications  which  can  be  used  in  different  platforms  now  use  two
factor authentication (2FA) to allow users to conveniently switch from one platform to another.
For example, when a user tries to login his gmail, it is not enough to enter correct password 
(first factor), it is also necessary to enter a PIN which is received by an SMS (second factor).
The aim of this project explaining the attacks against 2FA in IOS and Android devices and 
what can be the solutions. 
\end{abstract}
\end{comment}

\section{Introduction: The 2 Factors Authentication}

Two factor authentication (2FA) is a combination of 2 access control  in order to make it more robust to attacker.The general model is that to authenticate yourself to an online service you both need to provide an information you know (your password) and use a physical object you have. 
One of the most most used "physical object" is a phone or a smartphone
as large majority of people in today's world has one.
More specifically, it's the fact it's possible, through this phone, to send a message who will not transit by the (maybe compromised) PC 
who is used. your smartphone is used as an out of band channel.

While it exist other 2FA schemes we will only study here the ones using smartphones as they are the most frequent in practice. Moreover the attack will focus on SMS based scheme.
It make sense as it will discussed later that most of the other smartphones 2FA proposed by services an be passed by asking for SMS 2FA instead.

The precise scheme assumed here is the following: assuming we want to authentication to a service on a PC, after a successful password authentication, the user is required to enter (on the PC) a One-Time Passwords (OTP) send by SMS to the phone in order to fully authenticate to the service. 

This paper will present 2 attacks to break this authentication assuming a compromise PC and a sane smartphone. The former is to be used with android, the later on Ios.

This paper is essentially based on \href{http://fc16.ifca.ai/preproceedings/24_Konoth.pdf}{"How Anywhere Computing Just Killed Your Phone-Based Two-Factor Authentication" } by  Konoth, Radhesh Krishnan, Victor van der Veen, and Herbert Bos. \cite{Base}

%\subsection{2FA: 2 Factors Authentification}
\section{Key concepts}
Before describing precisely the attack, this section will explain and summarize the Key concept used by the 2 two attack,  The first subsection will describe what is synchronization and why it's a threat to the 2FA concept. the second describe the setup in which we assume the attack are performed: the phone is sane. the PC is compromised and is able perform a Man in the Browser attack.
\subsection{Synchronization}
Our use of informatics is mostly divided between our personal computers 
and smartphones. for this reason software manufacturer have sometime decided to implement synchronization processes between the 2 devices in order to make transition and general use of the 2 smoother, blurring the line between the two. This mean that the smartphone is less and less out of band with respect to the PC, which break the main assumption done by 2FA scheme using smartphone.

3 examples of such synchronization who will be used in our attack are the 
following:
\begin{description}
  \item[Google play remote install] it's possible from a PC via the Google play website to remotely install an application on our phone if both are logged on the same Google account. the only thing appearing on the phone afterward is the application icon in the app tray and an notification saying "<app\_name> as been successfully installed".
  
  \item[Apple Continuity] On recent version of Ios and macOs you can enable this setting to synchronize in clear, read and send your SMS from your mac.
  
  \item[Browser synchronization] Almost all of today's most popular Internet browser (i.e Firefox, Chrome or safari) propose synchronization between the mobile and desktop versions logged under same user account. it will sync history, bookmark and sometime currently open tabs.
\end{description}
\subsection{MitB: Man in the browser attack}
MitB is a type of attack who assume that the attacker has an entire control and view on the PC browser of the victim. Like a man in the middle attack, the attacker can see all data 
exchanged by the browser and server and can modify them (on the fly). it also can send and receive data in the name of the user. But unlike the former it has access to these data before they encrypted (or after they are decrypted) And has also modify browser r elated setting like bookmark and current open tabs URLs. In short powerful man in the browser attack can remotely perform the same actions has someone getting physical access to the browser as well as modify content of request and responses messages.

there are different way to do a MitB attack, using malware infecting the whole system, by API hooking or via malicious plugin. TO COMPLETE ?


\section{2FA Attack}
 this section describe attack to defeat 2FA under the following model. We assume that the PC of the customer is compromised and can perform man in the browser attack. The attacker try to authenticate on a service from the PC, It has get the password from the PC infection but the service request a second factor authentication by sending a one time password (OTP) by SMS to the victim smartphone who run is either under Adroid or Ios. 
This choice of model is motivated by the fact that 2FA is specially design to prevent authentication to services because of an infected PC 
\subsection{Android}
The principle to the attack is: via the MitB, Hijack an google session, via password or cookie stealing or else. Then Install from the web play-store an application with authorization to read SMS (authorization confirmed from the browser too). Then when an SMS is received it is forward via internet to the attacker or the compromised browser, who can successfully authenticate.

In order to succeed the attacker need to pass two defense setup by Google.

\paragraph{Bypassing Google boncer} Google remote install only allow to install application published on the Google play store. so for this attack, the attacker need successfully publish an SMS stealing app in the store. In order to do this it must bypass Google Boncer, an automated malware analysis tool deployed by Google on its store. Whenever an app in upload to the store, it's analyzed by Boncer, who perform static analysis (code inspection, ...) as well as dynamic analysis (sending request and action to the app and wait for malicious resulting behavior)

But Recent work CITE show that this defense is easy to bypass in various way. CITE. \cite{Base} is using an another clever way by opening a poorly protected of web window outside of screen from which it's possible to remotely execute malicious (Javascript) code, as this code is invisible to Google boncer and thus the app will be accepted.

Let's note that this possibility to load web content and let it execute some code on the app is an another intended feature made by Google called Web-View

\paragraph{Activate the app} When installed, an Android app can't be triggered by external event (such as RECEIVE\_SMS) until it has been explicitly open for the first time. So we need to trick the user to open the app from the phone. The 
first way is to give a clickbait title to the app so that the user will be tempted to open it when he sees the installation notification.

The second is to trigger its opening from the mobile browser by cliking a link

In both case just after this opening the can make itself disappear for homescreen and app tray, leaving little trace of it existence of the smartphone: only in the application list in the application part of the parameters menu. While still permanently running in the background.

\subsection{Ios}
Since 2015 Ios has forbid the application to read all notification or SMS without explicit authorization (or they will be rejected from entering the appstore). 
Since then, the previous attack does not work. 

However if the infected browser is on a Mac and Continuity is activated on the Iphone. the browser can still has access to to SMS in clear on the mac as soon as both are on the same LAN, which is likely to eventually occurs as they belong to the same person. 

So under these pretty likely to appears conditions, the 2FA authentication can be very easily bypassed on 2FA by an MitB attack.


\section{Discutions} 
This section will discuss practical feasibility of these attack and potential solution to defeat them on both platform. 
\subsection{It's not a exploit, it's a feature}
Before talking about feasibility we must emphases a particularity of these 2 attacks: After the initial infection of the PC no exploit, bug or hack are used. every tool we used as been purposely designed the use we makes of them by the developers ( at Google, Apple, microsoft or Mozilla). They propose way to reduce the air gap beetween the PC and the smartphone while a the same time security engineer rely on it too implement secure 2FA. And that's why fix to these attacks are complicated, there's nothing fix. developer must be aware that these functionality should impact 2FA and that it may need to be taken in account. And security engineering must be aware that these functionality are more and more present and that they break the assumption on which the security of their scheme relies.
\subsection{Android}
FESEAB

The truth is that a the solution against to this attack already exist but isn't fully deployed yet and can still be bypassed. Since Android Marshmallow, deployed in october 2015 the authorization model has change instead of granting all of them at installation time, they no need to be granted individually at run time when needed. With this new model authorization to read SMS can't be given from a remote browser. The only problem is that currently this new system work only for app compiled for Marshmallow and newer version, so it can be bypassed by purposely compile it for older version of android. The authorization granting scheme will be the old one, while still running on the newer versions. But its really likely that in the future it will be require for apps to be at least compiled for Marshmallow and the attack will be definitely defeated.
\subsection{Ios}
The Ios version of the atack is even simpler. 

\subsection{Other 2FA authentication on smartphone}
SMS 2FA are not the only one used to perform 2FA using smartphone. Majority of services also propose 2FA dedicate app (such as Google authenticator or Azure authenticator). And as on both Ios and Android application can't access each other data's (Application sandboxing technique) it defeat the presented attacks. But most of the service who implement 2FA this way also include SMS 2FA as a backup solution. If for some reason the user can't use or access the app (i.e lack of mobile network or buggy phone) it can still authenticate through SMS. So assuming the user has provide its phone number backup to the service. 2FA ussing dedicated app can still be defeat with the extra step of telling through the MitB that the user desire to use the backup solution to authenticate


\section{Conclusion}

The very soul of 2FA using smartphone is the assumption that there's an air barrier between your smartphone and your channel and is has been true for many year. But except for 2FA this is perceive more as a default than a desirable feature, and so manufacturer are implementing more and more synergies between these devices as an intended behavior. This need to be taken in account while implementing 2FA on smartphone. services must make sure that the channel they are using is really out of band with the user PC. And what's sure is that today 2FA through SMS is not secure anymore and should be avoided, even as a backup solution.

\bibliography{ref} 
\bibliographystyle{plain}

\end{document}  
